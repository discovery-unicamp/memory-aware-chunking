% Definição dos autores
\definechangesauthor[name={Edson Borin}, color=purple]{EB}
\definechangesauthor[name={Daniel Fonseca}, color=red]{DF}

% Borin (EB)
\newcommand{\EB}[1]{\comment[id=EB]{#1}} % Adiciona como comentário feito por EB
\newcommand{\EBH}[1]{\highlight[id=EB]{#1}} % Destaca o texto
\newcommand{\EBC}[2]{\EBH{#1} \comment[id=EB]{#2}} % Destaca e adiciona comentário
\newcommand{\EBADD}[1]{\added[id=EB]{#1}} % Adiciona texto
\newcommand{\EBRM}[1]{\deleted[id=EB]{#1}} % Remove texto
\newcommand{\EBRP}[2]{\replaced[id=EB]{#2}{#1}} % Substitui um texto pelo outro


% Edson's notes
% Daniel, mudei as macros por que não consegui adicionar comentários no rótulo de uma das tabelas com o modo anterior.
\renewcommand{\EB}[1]{{\textcolor{blue}{(EB: #1)}}}
\renewcommand{\EBH}[1]{{\hl{#1}}}
\renewcommand{\EBC}[2]{\EBH{#1}{\color{blue}(EB: #2)}}
\renewcommand{\EBADD}[1]{{\textcolor{blue}{#1}}}
\renewcommand{\EBRM}[1]{{\textcolor{lightgray}{(#1)}}}
\renewcommand{\EBRP}[2]{\EBRM{#1}\EBADD{#2}}
\newcommand{\EBRPD}[2]{#1 \EB{#1 $\rightarrow$ #2?}}

% Daniel (DF)
\newcommand{\DF}[1]{\comment[id=DF]{#1}} % Adiciona como coomentário feito por DF
\newcommand{\DFH}[1]{\highlight[id=DF]{#1}} % Destaca o texto
\newcommand{\DFC}[2]{\DFH{#1} \comment[id=DF]{#2}} % Destaca e adiciona comentário
\newcommand{\DFADD}[1]{\added[id=DF]{#1}} % Adiciona texto
\newcommand{\DFRM}[1]{\deleted[id=DF]{#1}} % Remove texto
\newcommand{\DFRP}[2]{\replaced[id=DF]{#2}{#1}} % Substitui um texto pelo outro
\newcommand{\DFTODO}[1]{\DFADD{TODO: #1}} % Adiciona um TODO com comentário