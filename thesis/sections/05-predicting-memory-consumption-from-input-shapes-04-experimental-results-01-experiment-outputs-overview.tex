\subsection{Experiment Outputs Overview}
\label{subsec:pmc-results-experiment-outputs-overview}

This subsection briefly outlines the three seismic operators examined (Envelope, \ac{GST3D}, and the 3D Gaussian Filter), underscores the main experimental objectives, and describes the key \ac{CSV} files generated during the final stages of the pipeline.
These \ac{CSV} artifacts are the foundation for the analyses presented in the following subsections.

\vspace{1em}
\noindent
\textbf{Operators and Experimental Goals.}
We investigated three commonly used seismic processing operators:
\begin{itemize}
    \item \emph{Envelope}: Computes instantaneous amplitude along seismic traces.
    \item \emph{\ac{GST3D}}: Highlights discontinuities or faults using structural tensors.
    \item \emph{3D Gaussian Filter}: Applies a smoothing operation across volumes to reduce high-frequency noise.
\end{itemize}
All three involve memory-intensive operations on potentially large \ac{3D} volumes.
Our primary goal was to build regression models that predict each operator’s peak memory usage as a function of shape parameters (inlines, xlines, and samples), enabling more accurate \ac{HPC} job submissions.

\vspace{1em}
\noindent
\textbf{Data Artifacts.}
During the final stage of the pipeline, five main \ac{CSV} outputs were generated:
\begin{enumerate}
    \item \emph{profile\_history.csv}: Stores time-series \ac{RSS} measurements and timestamps for each operator run. This file is useful for analyzing moment-to-moment memory fluctuations.
    \item \emph{profile\_summary.csv}: Summarizes peak memory usage, execution time, and statistical descriptors (means, standard deviations, minima, and maxima) per volume configuration.
    \item \emph{model\_metrics.csv}: Captures regression model performances (\ac{RMSE}, \ac{MAE}, $R^2$, and accuracy metrics), along with arrays of predictions and residuals.
    \item \emph{feature\_selection.csv}: Documents experiments that limit the predictor set to specific features or transformations to assess their importance in predicting memory usage.
    \item \emph{data\_reduction.csv}: Compares how models behave when trained on progressively smaller subsets of the full dataset, providing insight into data requirements for stable predictions.
\end{enumerate}

\vspace{1em}
\noindent
\textbf{Shape Configurations.}
The synthetic seismic volumes encompassed a broad range of \ac{3D} shapes, from $100 \times 100 \times 100$ up to sizes near the system’s memory limits.
By spanning both small and large volumes, we could characterize memory usage trends across diverse problem scales.
Each operator was run on each shape variant, yielding a comprehensive grid of memory and runtime measurements.