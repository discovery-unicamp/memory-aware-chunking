\section{Materials and Methods}
\label{sec:pmc-materials-and-methods}

This section explains the complete methodology for gathering memory-consumption data, generating shape-based features, and training predictive models.
The overall pipeline includes four core phases.
(i) Data Generation,
(ii) Memory Profiling,
(iii) Result Collection,
and (iv) Analysis.
These phases ensure consistent, reproducible, and high-quality measurements, which support machine learning models that learn memory usage from shape attributes.

\vspace{1em}
\noindent
\textbf{Phase 1: Data Generation.}
The experiment begins with synthetic seismic datasets.
In a typical seismic application, three dimensions describe the data.
\emph{inlines}, \emph{xlines}, and \emph{samples}.
The script systematically enumerates shape configurations by varying these dimensions from an \texttt{INITIAL\_SIZE} to a \texttt{FINAL\_SIZE} with increments of \texttt{STEP\_SIZE}.
For example, when the initial size equals 100, the final size equals 600, and the step size equals 100, the script generates volumes of sizes
\(\{\,(100,100,100),\,(100,100,200),\dots,(600,600,600)\}\)
and writes them to \ac{SEG-Y}~\cite{barry1975segy} files.
These synthetic datasets mimic actual seismic data and span diverse volumes to reveal how memory grows with increasing dimension sizes.

\vspace{1em}
\noindent
\textbf{Phase 2: Memory Profiling.}
A shell script schedules container-based jobs to run seismic operators on each generated dataset.
Chapter~\ref{ch:measuring-memory-consumption} describes the \ac{HPC}-like constraints and the rationale for isolating each job to reduce interference.
Accordingly, the script starts \ac{dind} containers pinned to specific \ac{CPU} cores through \texttt{--cpuset-cpus}.
This binding ensures comparable runs.
The script then processes three main operators:

\begin{itemize}
    \item \emph{Envelope}~\cite{taner1979complex}: A classic seismic attribute that computes instantaneous amplitude for each trace.
    \item \emph{\ac{GST3D}}~\cite{bigun2004recognition}: A structural attribute designed to highlight seismic faults or discontinuities.
    \item \emph{3D Gaussian Filter}~\cite{gonzalez2002digital}: A spatial smoothing operator for seismic volumes.
\end{itemize}

Each operator appears as a Docker image defined in the \texttt{Dockerfile}.
Each container execution uses a Python script to load and process input data.
The \texttt{traceq} library records \ac{RSS} over time, as \EBRP{shown}{discussed} in Chapter~\ref{ch:measuring-memory-consumption}.
On completion, a \texttt{.prof} file stores the memory timeline and associated metadata, including timestamps and shape parameters.
This procedure captures peak memory usage accurately.

\vspace{1em}
\noindent
\textbf{Executing Operators and Gathering Profiles.}
Listing~\ref{lst:exp_sh} shows portions of the shell script.
The script orchestrates the sequence.
(i) create Docker volumes,
(ii) run a Python script to generate the synthetic datasets,
(iii) for every generated \ac{SEG-Y} file, execute Envelope, \ac{GST3D}, or Gaussian Filter,
(iv) run a Python script to parse and consolidate measurements,
and (v) finalize executing another Python script for exploratory data analysis.
Each run deposits output \ac{CSV} files and profiles in designated \texttt{OUTPUT\_DIR} subdirectories.

\vspace{1em}
\begin{lstlisting}[style=bashstyle,caption={Excerpts from \texttt{experiment.sh}~\cite{delucca2025experiment2script} that orchestrate Docker-based runs. Variables like \texttt{DATASET\_FINAL\_SIZE} and \texttt{DATASET\_STEP\_SIZE} define shape ranges for dataset generation.}, label={lst:exp_sh}]
#!/usr/bin/env sh
TIMESTAMP="$(date +%Y%m%d%H%M%S)"
CPUSET_CPUS="0"
DATASET_FINAL_SIZE="800"
DATASET_STEP_SIZE="50"

echo "Generating input data..."
docker run \
  --cpuset-cpus=0 \
  -v "${DIND_VOLUME_NAME}:/var/lib/docker:rw" \
  ...
  --env EXPERIMENT_COMMAND="generate_data.py" \
  --env EXPERIMENT_ENV="\
    -e OUTPUT_DIR=/experiment/out \
    -e FINAL_SIZE=${DATASET_FINAL_SIZE} \
    -e STEP_SIZE=${DATASET_STEP_SIZE} \
  " \
  docker:28.0.1-dind \
  "/workspace/experiment.sh"

echo "Collecting memory profile for Envelope..."
for file in "${OUTPUT_DIR}/inputs"/*.segy; do
  filename=$(basename "$file" .segy)
  docker run \
    --cpuset-cpus=0 \
    ...
    --env EXPERIMENT_COMMAND="collect_memory_profile.py" \
    --env EXPERIMENT_ENV="\
      -e ALGORITHM=envelope \
      -e INPUT_PATH=/experiment/out/inputs/${filename}.segy \
    " \
    docker:28.0.1-dind \
    "/workspace/experiment.sh"
done
...
\end{lstlisting}

\noindent
Table~\ref{tab:env_vars} summarizes relevant environment variables used by these scripts.
They control dataset sizes, repetition counts, and output paths.

\begin{table}[htbp]
    \centering
    \caption{Key environment variables for the experimental scripts.
    Defaults appear in parentheses.
    \vspace{1em}}
    \label{tab:env_vars}
    \begin{tabular}{ll}
        \hline
        \textbf{Variable}             & \textbf{Description}                                                                        \\
        \hline
        \texttt{DATASET\_FINAL\_SIZE} & Final dimension size (e.g., 800)                                                            \\
        \texttt{DATASET\_STEP\_SIZE}  & Step size between dimension increments (e.g., 50)                                           \\
        \texttt{EXPERIMENT\_N\_RUNS}  & Number of repeated runs per shape (e.g., 30)                                                \\
        \texttt{CPUSET\_CPUS}         & \ac{CPU} core pinning for Docker containers (e.g., 0)                                       \\
        \texttt{OUTPUT\_DIR}          & Directory for output profiles, logs, and \texttt{\ac{CSV}} files                            \\
        \texttt{ALGORITHM}            & Selected operator (e.g., \texttt{envelope}, \texttt{\ac{GST3D}}, \texttt{gaussian\_filter}) \\
        \hline
    \end{tabular}
\end{table}

\vspace{1em}
\noindent
\textbf{Phase 3: Results Collection and Feature Construction.}
This phase consists in evaluating the \texttt{.prof} files output by \texttt{traceq}, extracting peak memory usage, and associating it with the input shape.
The script identifies \EBC{inlines, xlines, and samples}{estes três termos aparecem com formatação distintas ao longo do texto - normal, itálico e negrito. Não ficou claro para mim por que eles são destacados de forma distinta. Seria bom padronizar a forma de formatar estes termos.} from filenames, notes the operator, retains the maximum time-series \ac{RSS} value, and compiles a consolidated dataframe.
It also derives extended features such as total volume
(\(\text{inlines} \times \text{xlines} \times \text{samples}\)),
logarithmic transforms
(e.g., \(\log_2(\text{volume})\)),
surface area,
dimension ratios,
and polynomial expansions.
These features capture various ways shape correlates with peak memory usage.
Listing~\ref{lst:results_py} shows how this phase generates and transforms these features.

\begin{lstlisting}[style=pythonstyle,
    caption={Feature extraction excerpt from \texttt{collect\_results.py}~\cite{delucca2025experiment2resultscollection}. Additional derived metrics capture polynomial or ratio-based growth.},
    label={lst:results_py}]
df_features["volume"] = (
    df_features["inlines"]
    * df_features["xlines"]
    * df_features["samples"]
)

df_features["diagonal_length"] = np.sqrt(
    df_features["inlines"]**2
    + df_features["xlines"]**2
    + df_features["samples"]**2
)

df_features["surface_area"] = 2 * (
    df_features["inlines"] * df_features["xlines"]
    + df_features["inlines"] * df_features["samples"]
    + df_features["xlines"] * df_features["samples"]
)

df_features["log_inlines"] = np.log2(df_features["inlines"])
df_features["log_volume"] = np.log2(df_features["volume"])
...
\end{lstlisting}

\vspace{1em}
\noindent
\textbf{Phase 4: Model Training and Analysis.}
The script from phase 3 trains multiple regression models using \texttt{inlines}, \texttt{xlines}, \texttt{samples}, and derived features as predictors.
Peak memory usage acts as the target variable.
A consistent 80/20 training/test split ensures fair comparisons.
Envelope, \ac{GST3D}, and Gaussian Filter runs each receive separate analyses.
Performance metrics include \ac{RMSE}~\cite{hyndman2006}, \ac{MAE}~\cite{willmott2005mae}, $R^2$~\cite{draper1998applied}, and an \EBC{accuracy score that measures the fraction of predictions within a specified relative error}{Tem que descrever como este score é calculado}.
An \texttt{Optuna}-based~\cite{akiba2019optuna} parameter search can emphasize safe upper-bound estimates in \ac{HPC} contexts where underestimations produce critical failures.
The following models form the basis of each comparison:

\begin{itemize}
    \item \textbf{Linear Regression, Polynomial Regression}~\cite{hastie2009elements}\\
    These models create foundational baselines to capture the relationship between features and memory usage.
    Linear regression indicates whether memory usage scales directly with \EBC{trace or sample counts}{não seria mais preciso dizer "any of the features, suach as trace or smaple counts"?}.
    Polynomial regression extends linear models with higher-order terms to account for mild nonlinearities.
    Caution is required since purely linear fits underfit complex relationships, and high-degree polynomials risk overfitting.

    \item \textbf{Decision Trees}~\cite{breiman1984classification}, \textbf{Random Forest}~\cite{breiman2001random}, \textbf{\ac{XGBoost}}~\cite{chen2016xgboost}\\
    \EB{Lembre-se de revisar a versão final para identificar e corrigir eventuais invasões de margem, como acima.}
    These tree-based methods automatically model complex feature interactions and nonlinearities.
    Decision Trees provide easy interpretability and can indicate shape thresholds that strongly affect memory.
    Random Forest aggregates multiple trees to reduce variance and enhance generalization.
    \ac{XGBoost} incrementally refines an ensemble of weak learners, which often results in higher accuracy on structured data.
    Hyperparameter tuning (e.g., tree depth or learning rate) is essential to avoid overfitting. \EB{Seria bom adicionar uma referência para dar suporte a este comentário.}

    \item \textbf{Neural Networks (\ac{MLP})}~\cite{rumelhart1986learning}\\
    \ac{MLP} architectures can approximate intricate functions when provided sufficient capacity.
    This property can reveal subtle relationships between shape parameters and memory requirements.
    These models often demand larger datasets, rigorous hyperparameter optimization, and careful overfitting control. \EB{Seria bom adicionar uma referência para dar suporte a este comentário.}

    \item \textbf{Gaussian Processes}~\cite{rasmussen2006gaussian}\\
    This approach offers both mean predictions and uncertainty estimates.
    Uncertainty bounds are especially valuable in \ac{HPC} scenarios where underestimating memory can cause abrupt job failures.
    Large datasets pose computational challenges because of the \(O(n^3)\) scaling, and kernels must be selected carefully. \EB{Seria bom adicionar uma referência para dar suporte a este comentário.}

    \item \textbf{Bayesian Ridge Regression}~\cite{bishop2006pattern}\\
    This extension of linear models applies Bayesian inference to estimate coefficients and yields robust predictions with uncertainty quantification.
    HPC workloads benefit from these uncertainty estimates when resource over-provisioning is less harmful than running out of memory.
    Polynomial or interaction terms might still be necessary to represent nonlinear growth patterns. 
\end{itemize}

The script final Python script then visualizes memory distributions, residual plots, feature evaluations, and data-reduction experiments.
This exploration reveals each model’s generalization capacity, indicates which features add minimal value, and evaluates the number of training examples that the predictive framework requires.
Partial dependence plots and tree-based model interpretations expose how inlines, xlines, and samples exert the highest influence on memory usage.

\vspace{1em}
\noindent
\textbf{Technical Considerations.}
\begin{itemize}
    \item \emph{Containerization and Reproducibility:}
    Each stage runs in a Docker container with pinned versions of Python packages.
    This standardized environment preserves consistent results by preventing version drift.
    \item \emph{\ac{CPU} Affinity:}
    The \texttt{--cpuset-cpus=0} parameter restricts execution to a single core and reduces background noise.
    Multiple runs per dataset shape (\texttt{EXPERIMENT\_N\_RUNS}) further \EBC{stabilize measurements.}{accounts for stochatic variance?}
    \item \emph{Parallelization:}
    The Docker orchestration scales to multiple \ac{CPU} cores, but the design imposes single-core experiments for repeatability.
    Actual \ac{HPC} clusters manage concurrency at the scheduler level, but the core principle of shape-driven memory profiling remains applicable.
    \item \emph{Cache Handling:}
    Chapter~\ref{ch:measuring-memory-consumption} recommends dropping Linux memory caches to eliminate carry-over effects.
    This option appears in the scripts to ensure a clean environment for memory measurements.
\end{itemize}

\vspace{1em}
\noindent
Figure~\ref{fig:pmc_datapipeline} illustrates the entire workflow.
The process generates synthetic \ac{SEG-Y} volumes, runs containerized operators for memory profiling, and consolidates results into a final dataset that links shape features to peak memory usage.
Regression models then learn to map input dimensions to memory demand.
Section~\ref{sec:pmc-results} elaborates on these experimental outcomes and compares the predictive accuracy of the considered models.

\begin{figure}[htbp]
    \centering
    \resizebox{\textwidth}{!}{
        \begin{tikzpicture}[
            font=\sffamily\small,
            phase/.style={
                draw=black,
                thick,
                fill=gray!10,
                rectangle,
                rounded corners,
                drop shadow,
                align=center,
                minimum width=3.8cm,
                inner sep=0.4cm
            },
            box/.style={
                draw=black,
                thick,
                fill=white,
                rectangle,
                rounded corners,
                align=center,
                minimum width=3.8cm,
                minimum height=1.1cm
            },
            arrow/.style={
                ->,
                thick
            },
            node distance=0.4cm and 0.8cm
        ]

            % Phase 1: Data Generation
            \node[phase] (phase1) {Phase 1:\\Data Generation};
            \node[box, below=of phase1] (p1input) {\textbf{Input}\\Shape parameters\\(inlines, xlines, samples)};
            \node[box, below=of p1input] (p1process) {\textbf{Process}\\Generate synthetic\\SEG-Y volumes};
            \node[box, below=of p1process] (p1output) {\textbf{Output}\\Synthetic data files};

            \draw[arrow] (p1input) -- (p1process);
            \draw[arrow] (p1process) -- (p1output);

            % Phase 2: Memory Profiling
            \node[phase, right=3cm of phase1] (phase2) {Phase 2:\\Memory Profiling};
            \node[box, below=of phase2] (p2input) {\textbf{Input}\\Synthetic SEG-Y files};
            \node[box, below=of p2input] (p2process) {\textbf{Process}\\Container-based\\operators (traceq)};
            \node[box, below=of p2process] (p2output) {\textbf{Output}\\.prof files};

            \draw[arrow] (p2input) -- (p2process);
            \draw[arrow] (p2process) -- (p2output);
            \draw[arrow] (p1output.east) -- ++(0.7,0) |- (p2input.west);

            % Phase 3: Feature Aggregation
            \node[phase, right=3cm of phase2] (phase3) {Phase 3:\\Aggregation \& Feature\\Construction};
            \node[box, below=of phase3] (p3input) {\textbf{Input}\\.prof files + shape info};
            \node[box, below=of p3input] (p3process) {\textbf{Process}\\Parse peak usage\\Derive features};
            \node[box, below=of p3process] (p3output) {\textbf{Output}\\Feature dataset};

            \draw[arrow] (p3input) -- (p3process);
            \draw[arrow] (p3process) -- (p3output);
            \draw[arrow] (p2output.east) -- ++(0.7,0) |- (p3input.west);

            % Phase 4: Model Training
            \node[phase, right=3cm of phase3] (phase4) {Phase 4:\\Model Training \& Evaluation};
            \node[box, below=of phase4] (p4input) {\textbf{Input}\\Feature dataset};
            \node[box, below=of p4input] (p4process) {\textbf{Process}\\Train regression\\(XGBoost, NN, etc)};
            \node[box, below=of p4process] (p4output) {\textbf{Output}\\Memory usage\\predictions};

            \draw[arrow] (p4input) -- (p4process);
            \draw[arrow] (p4process) -- (p4output);
            \draw[arrow] (p3output.east) -- ++(0.7,0) |- (p4input.west);

        \end{tikzpicture}
    }
    \caption{Schematic of the memory prediction pipeline. It includes synthetic data generation, containerized profiling, feature extraction, and regression model training.}
    \label{fig:pmc_datapipeline}
\end{figure}