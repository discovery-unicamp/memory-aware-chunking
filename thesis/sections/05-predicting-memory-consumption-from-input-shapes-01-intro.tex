\section{Introduction}
\label{sec:pmc-introduction}

\ac{HPC} environments require users to specify computing resources, especially memory, before a job begins execution.
Common \ac{HPC} job schedulers, such as Slurm~\cite{yoo2003slurm} or \ac{PBS}~\cite{henderson1995pbs}, enforce fixed memory reservations at submission time.
This enforcement compels users to estimate a job's memory footprint in advance.
That requirement creates a fundamental challenge: users must predict memory needs \emph{a priori}, often without the benefit of trial runs.
Setting the memory request too low causes out-of-memory errors~\cite{bailey2005,hovestadt2003}, leading to lost computing time and queued tasks.
Setting it too high wastes memory and reduces overall system utilization because nodes retain unused allocations.
Accurate memory estimation therefore remains crucial for reliability and efficiency in \ac{HPC} resource management.

Most users rely on guesswork to determine memory reservations.
Many \ac{HPC} users adopt trial-and-error methods~\cite{ncsu_memory_usage}\EB{Tente substituir por uma referência a um artigo publicado.}, iteratively adjusting memory limits across repeated job submissions until one run completes without errors.
This approach leads to inefficiency and significant risk.
Each failed attempt wastes wall-clock time and computational cycles, while each successful run with excessive memory allocations deprives other jobs of available resources.
Over-provisioning represents a common workaround but leaves systems underutilized.
That trial-and-error paradigm reduces overall throughput and burdens users with resource-tuning responsibilities instead of scientific exploration.

Predictive modeling for memory allocation offers a clear path to improvement.
Models can forecast a job's memory demand using readily available workload features such as input data size, dimensionality, algorithmic parameters, or usage patterns from similar jobs.
These forecasts reduce guesswork and limit the need for iterative resubmissions.
More accurate pre-runtime memory predictions enable users to request memory more precisely, mitigating failure risks and alleviating resource waste~\cite{tanash2021ampro}.

The content in this chapter prioritizes \EBC{tensor-based}{tensor-based é usado aqui, mas array-based em usado mais à frente. Seria bom padronizar o uso destes termos. Além disso, tensor-based, tensor-centric e outros termos baseados em tensores aparecem aqui mas não foram definidos. Seria bom definí-los no primeiro uso do termo.} workloads, which exhibit strong correlations between memory usage and input shape.
Numerous scientific and data-centric applications center on large multidimensional arrays, known as tensors.
Examples include seismic wave simulations on \ac{3D} grids, high-resolution image analysis in computer vision, and large-scale numerical linear algebra routines in scientific computing.
Memory consumption in these scenarios often scales according to array dimensions, because storage for input data structures and intermediate results typically dominates memory usage.
For example, doubling the resolution of an input image in each dimension roughly quadruples its memory footprint~\cite{stackoverflow_memory_inv}, while in deep neural networks increasing either input size or batch size similarly amplifies memory usage due to larger activation maps and additional computation~\cite{dell_3dunet_memory}.
These relationships underscore the crucial role of input shape in determining memory demands for tensor-centric workloads.
A predictive model that captures how memory scales with shape parameters can then offer robust estimates for new input configurations.

Several research directions have explored memory usage estimation and prediction in related fields.
In the \ac{HPC} domain, historical job data from systems like Slurm often enable machine learning models to predict resource consumption.
These models exploit features such as job metadata and input sizes to forecast memory requirements~\cite{yoo2003slurm}.
That approach demonstrates that ample historical data can support reliable predictions for recurring workloads.
In high-level numerical computing, analytical modeling of memory for array operations provides another possibility~\cite{cornell_memory_workshop}.
Matrix multiplications, convolutions, and other array-based methods can be studied to deduce peak memory usage without full execution.
Deep learning frameworks take a similar route to anticipate a network's memory footprint.
Layer-by-layer analysis of network architectures and input tensor dimensions reveals how memory usage accumulates~\cite{gao2020, dell_3dunet_memory}.
Profiling and simulation tools further assist deployment decisions by projecting memory usage before runtime~\cite{tanash2019}.

The primary objective of the present chapter is to estimate memory usage for new jobs given their input shapes, rather than to yield an exact byte-level measurement.
This emphasis on practical accuracy aims to provide a sufficiently precise estimate of peak memory demand to guide resource allocation.
Slight discrepancies arise due to overheads and system factors, but a well-calibrated predictor avoids severe underestimates that result in \ac{OOM} failures.
Focusing on a reliable upper-bound estimate allows users to reserve memory with greater confidence.
Narrowing the gap between requested and consumed memory improves performance over purely heuristic-based approaches and alleviates the risks and inefficiencies commonly associated with guesswork.