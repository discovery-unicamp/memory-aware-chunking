%\section{Conclusion}
%\label{sec:pmc-conclusion}
%
%•	Feasibility of Shape-Based Memory Prediction: Our investigation confirms that predicting memory consumption from input shapes is not only feasible but can be highly accurate for tensor-based workloads. Even using relatively straightforward features (dimensions, their products) and general-purpose ML models, we achieved predictions that are within a few percent of actual usage for both synthetic tests and real seismic data. This validates the core hypothesis of this chapter: for data-intensive computations where the algorithm doesn’t drastically change behavior based on data content, input size is the primary driver of memory demand. We have shown that a model can learn this relationship, essentially automating what an expert might deduce (e.g., “if I double the grid size, memory will roughly double”) with far greater precision. In practical terms, this means developers and HPC users can rely on a predictive system to estimate memory needs instead of guesswork.
%•	Benefits for Seismic Workflows: In the seismic domain, datasets are large and processing them is memory-intensive. By deploying a memory predictor trained as we described, geoscientists can plan their computational jobs more efficiently. For example, before launching a seismic inversion or attribute extraction on a new survey, one can input the survey’s dimensions into the predictor to get an estimate of peak memory required. This helps in choosing an appropriate computing node or cluster configuration (ensuring enough memory is available) and in reserving just the right amount of memory. The outcome is fewer failed jobs due to OOM (out-of-memory) errors and less wasted resources from over-allocation. It also enables what-if analyses: users can simulate how memory demand grows if, say, they increase the resolution of their grid or the size of a processing batch, guiding decisions about data resizing or algorithm parameter tuning. For the F3 dataset example, our model could have immediately informed the user that on a 256 GB node, the job will fit with room to spare, whereas on a 64 GB node it would likely exceed available memory – avoiding an attempt that would inevitably crash. Such foresight is extremely valuable when pipeline execution time is measured in hours or days; it prevents lost time due to memory errors.
%•	Cluster Resource Management Improvements: More broadly, integrating these predictive models into cluster scheduling systems can improve overall HPC operations. If each job can come with a more accurate memory requirement estimate (either provided by an offline tool or an automated script that uses input metadata to predict), the scheduler can pack jobs more optimally and reduce idle memory. This translates to higher throughput and utilization of expensive HPC resources (https://pmc.ncbi.nlm.nih.gov/articles/PMC9906793/#:~:text=the%20time,consumption%20of%20the%20HPC%20resources). As noted in related work, when users specify resources more accurately, it reduces queue wait times and increases system efficiency (https://pmc.ncbi.nlm.nih.gov/articles/PMC9906793/#:~:text=While%20these%20resource%20requirements%20are,waiting%20time%20for%20submitted%20jobs).
%Our approach could complement such systems as a plugin: for instance, a submission script could call a “memory predictor” given the input file dimensions before filling in the resource request. Some HPC centers are already exploring machine learning to assist users in resource selection (https://pmc.ncbi.nlm.nih.gov/articles/PMC9906793/#:~:text=,waiting%20time%20for%20submitted%20jobs),
%and a shape-based predictor can be one piece of that puzzle, especially for regular data-processing jobs. Additionally, from a power and cost perspective, avoiding overallocation has tangible benefits (power savings from not needlessly reserving memory and spinning up large nodes) (https://pmc.ncbi.nlm.nih.gov/articles/PMC9906793/#:~:text=the%20time,consumption%20of%20the%20HPC%20resources)
%
%	•	\textbf{Generality to Other Domains:} While our case study centered on a seismic processing algorithm, the methodology and benefits extend to other high-dimensional data processing tasks. Any application that primarily handles large arrays or matrices could use a similar predictive model. For example, in climate modeling or CFD simulations, one could predict memory from the grid resolution; in bioinformatics, from the genome length or number of sequences being processed; in deep learning, from the model architecture and input image size (in fact, researchers have built models to estimate memory usage of neural networks given layer configurations (https://infohub.delltechnologies.com/sv-se/l/memory-consumption-modeling-of-deep-learning-workloads/case-study-application-3d-u-net/#:~:text=intermediate%20tensors%20,depends%20on%20four%20key%20parameters)
%The common theme is that \textbf{a few key parameters define the problem size}, and thus memory. By training models on those parameters, one can obtain quick estimates without running the full application. This is akin to performance modeling efforts in HPC, but focused on memory rather than runtime. Our work demonstrates a blueprint for building such models with modest effort and data.
%
%
%	•	\textbf{Practical Deployment Considerations:} To implement this in practice, one would integrate the trained model into the workflow. The model itself (especially if using a simple regressor or tree ensemble) can be saved and embedded in a small CLI tool or library. Given an input shape (perhaps read from the data header or inferred from input), the tool outputs an estimated peak memory. Users might still include a safety margin (e.g., request 10% above the estimate) to account for any unforeseen overhead or to satisfy cluster integer memory units. Since our results indicated robust performance, that margin can be small. It’s worth noting that if the range of input sizes changes (say, even bigger data than seen before), the model should be retrained or at least updated with new samples – an ongoing learning process. However, gathering a few new sample points (by profiling the new extreme cases) and retraining is far less onerous than continuously guessing and checking. The cost of profiling is just a few runs of the program (which one might do in testing anyway), and after that the model spares you from surprises.
%
%
%	•	\textbf{Limitations and Future Work:} Our approach deliberately simplifies the problem by assuming a single application and focusing only on input shape. In more complex scenarios, memory usage could also depend on other factors: e.g., algorithmic parameters (tolerance settings that change iteration counts), data values (sparse vs dense data might use memory differently), or concurrent workloads on the same node. For our seismic use case, these were not significant factors, but in general, expanding the feature set might further improve predictions if needed. Future work could explore \textbf{multi-variable models}, incorporating additional inputs like algorithm flags or even real-time performance counters. Another avenue is to create \textbf{dynamic predictors} that predict not just the peak but the timeline of memory usage, enabling optimizations like when to start transferring data out to disk, etc. Moreover, applying this method to \textit{runtime} prediction alongside memory would give a full picture of resource needs – something HPC schedulers would highly value (https://www.hpcwire.com/solution_content/ibm/cross-industry/a-crystal-ball-for-hpc/#:~:text=A%20Crystal%20Ball%20for%20HPC,runtime%20required%20for%20their%20workloads)
%
%
%	\textbf{Conclusion:} In conclusion, this chapter demonstrated a successful case of using machine learning to predict peak memory consumption from input shapes for computational workloads. We transformed a traditionally heuristic, trial-and-error task into a data-driven modeling task, and achieved high accuracy and reliability. The predictive models effectively serve as a “crystal ball” for memory usage, allowing us to anticipate resource demands before execution. The implications are substantial: with such tools, scientists and engineers can run large-scale computations with greater confidence and efficiency. No longer must one “overestimate by a large margin just in case” (https://pmc.ncbi.nlm.nih.gov/articles/PMC9906793/#:~:text=While%20these%20resource%20requirements%20are,waiting%20time%20for%20submitted%20jobs)
%– instead, one can request resources smartly, guided by predictions. This fosters better usage of HPC infrastructure and smoother user experience. The success with the seismic data example paves the way for adopting similar strategies in other domains dealing with big data and arrays. Ultimately, predictive memory modeling contributes to the broader goal of \textbf{autonomous resource management} in computing – where systems can self-tune and allocate resources optimally based on learned models, minimizing human guesswork and error. The next chapter will build on these insights, possibly exploring runtime performance prediction or integrating our memory predictor into a live scheduling system, to further illustrate the power of predictive modeling in HPC environments.