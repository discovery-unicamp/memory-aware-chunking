\subsection{Conclusion}
\label{subsec:pmc-conclusion}

This chapter explored the relationship between seismic input shapes and operator-specific memory consumption, focusing on three major processing routines—Envelope, \ac{GST3D}, and Gaussian Filter—and detailed how varying sample sizes, selected features, and modeling pipelines affect predictive accuracy.
Several salient findings emerged:

\begin{itemize}
    \item \textbf{Linear Volume Dependence and Operator Sensitivity.}
    All operators demonstrated a near-linear escalation in peak \ac{RAM} usage with respect to total input volume (\( \text{inlines} \times \text{xlines} \times \text{samples} \)).
    \ac{GST3D} remained the most sensitive, displaying a steeper volume-based slope compared to Envelope and Gaussian Filter.
    Envelope and Gaussian Filter, by contrast, showed comparatively gentler linear relationships, suggesting that their computational kernels rely less on large intermediate buffers.

    \item \textbf{Feature Selection and the Predominance of Volume.}
    Systematic removal of shape-derived features (e.g., diagonal length, surface area, ratio transformations) underscored that \emph{volume} alone captures most of the variance in peak memory usage.
    Even advanced transformations (logarithmic or polynomial terms) contributed only marginal improvements once volume was included.
    \ac{GST3D} occasionally benefited from additional parameters (e.g., diagonal length), but not enough to outweigh the straightforward predictive power of volume.

    \item \textbf{Model Performance and Robustness.}
    Across nine regression approaches, simpler or regularized methods—like \ac{Linear Regression}, Elastic Net, and decision-tree ensembles—performed well, achieving \(R^2 \approx 0.99\) or higher in many configurations.
    Specifically:
    \begin{itemize}
        \item \textbf{Envelope}: Gradient Boosting offered top-level accuracy, though multiple models clustered in the same performance range.
        \item \textbf{\ac{GST3D}}: Decision Trees (and some ensemble variants) captured memory usage reliably but exhibited greater sensitivity to missing data in the mid-volume range.
        \item \textbf{Gaussian Filter}: \ac{Linear Regression} proved sufficient, reinforcing that this operator’s memory footprint aligns closely with volume.
    \end{itemize}
    All methods exhibited mild to moderate right-skew in their residual distributions, yet \ac{RMSE} and \ac{MAE} remained low once mid- and upper-bound volumes were adequately represented.

    \item \textbf{Subsampling and Data Requirements.}
    Pruning the dataset to 30–40 shape configurations preserved high $R^2$ values, indicating that modest sampling—particularly anchored at small and large volumes—enables accurate modeling.
    \ac{GST3D} showed slightly larger degradation at smaller sample counts, reflecting its more complex intermediate data allocations.
    By contrast, Envelope and Gaussian Filter sustained robust predictions with moderate data reduction.
\end{itemize}

Overall, Chapter~\ref{sec:pmc-results} establishes that:
\begin{enumerate}
    \item \emph{Volume is the primary explanatory feature} for predicting peak memory usage, with additional variables offering only minor gains.
    \item Simple or linear-in-spirit regressors often suffice to model Envelope and Gaussian Filter, while \ac{GST3D} can benefit from tree-based methods if data coverage in the mid-volume range is adequate.
    \item A sample size of approximately 30–40 shape configurations provides a practical lower bound for stable memory prediction, making data collection efforts more tractable for real-world \ac{HPC} scenarios.
\end{enumerate}

With these findings, we have a clearer framework for implementing memory-aware chunking strategies: one can rely on volume-based models (potentially augmented by minimal shape features) and still maintain accurate peak \ac{RAM} consumption estimates, even with relatively small training sets.
The subsequent chapter applies these insights to optimize data-parallel chunking and scheduling decisions for large-scale seismic workloads.