\section{Experimental Setup}
\label{sec:pmc-experimental-setup}

The experiment followed a pipeline composed of seismic data generation, containerized processing with memory profiling, and consolidation of results for modeling.
Each step was fully automated to ensure consistency, reproducibility, and scalability.
The subsections below detail the execution environment, data generation, operator processing, memory profiling, and data consolidation strategies.

\subsection{Execution Environment and Workflow Automation}
\label{subsec:pmc-execution-environment-and-workflow-automation}

All experiments executed on a single \ac{HPC} node at the \ac{UNICAMP} Discovery Labs, equipped with an Intel\textregistered\ Xeon\textregistered\ Silver 4310 processor, 256~\ac{GB} of \ac{RAM}, and two \ac{RTX}~A6000 \ac{GPU}s. Despite the presence of \ac{GPU} resources, all memory profiling runs relied exclusively on \ac{CPU} execution to ensure comparability across input shapes.

A Python-based container environment managed via Docker provided reproducibility and isolation.
Docker containers pinned to specific \ac{CPU} cores (\texttt{–cpuset-cpus=0}) maintained execution consistency and minimized process interference.

A single shell script, \texttt{scripts/experiment.sh}\cite{delucca2025experiment2script}, orchestrated the full workflow, including:
\begin{enumerate}
    \item Creation of Docker volumes for intermediate and output data storage.
    \item Generation of synthetic \ac{3D} seismic volumes, as described in Section\ref{sec:pmc-materials-and-methods}.
    \item Execution of seismic operators within containers, one run per input shape configuration.
    \item Memory profiling using \texttt{TraceQ}, which captured \ac{RSS} measurements over time.
    \item Extraction and consolidation of peak memory values, shape metadata, and execution logs into structured \texttt{.csv} files.
\end{enumerate}

The script looped over all permutations of \texttt{(inlines, xlines, samples)}, parsing \texttt{TraceQ} outputs and joining them with shape metadata.
The final output consisted of a consolidated dataset for model fitting.

\subsection{Synthetic Data Generation and Real-Data Validation}
\label{subsec:pmc-synthetic-data-generation-and-real-data-validation}

\paragraph{Synthetic Volumes.}
A comprehensive enumeration of \ac{3D} input shapes defined the experimental space.
Input volumes ranged from $100$ $\times$ $100$ $\times$ $100$ to shapes approaching system memory limits.
Each synthetic volume was written in the \ac{SEG-Y} format to maintain consistency with the seismic processing pipeline.
This method enabled a controlled exploration of both common and extreme input configurations.

\paragraph{Netherlands F3 Dataset~\cite{f3dataset}.}
The standard Netherlands F3 seismic volume ($651$ $\times$ $951$ $\times$ $462$) served as a real-data validation benchmark.
Memory usage predictions from the trained models were compared against actual measurements on this dataset, enabling evaluation of model generalization to real-world data.

\subsection{Containerized Seismic Operators and Memory Profiling}
\label{subsec:containerized-seismic-operators-and-memory-profiling}

Each shape configuration was processed in a dedicated Docker container with exclusive \ac{CPU} pinning to reduce runtime variance.
Three memory-intensive seismic operators were evaluated: \emph{Envelope}, \emph{\ac{GST3D}}, and a \emph{3D Gaussian Filter}.
These operators were treated as black-box transformations that received seismic volumes and returned transformed data, with no internal modifications or visibility into their implementation.

The container startup routine activated \texttt{TraceQ}, which monitored the Python process and recorded \ac{RSS} snapshots at regular intervals.
Upon operator completion, \texttt{TraceQ} wrote a \texttt{.prof} file containing the memory usage timeline.
This design minimized noise from external processes and ensured that each memory profile reflected the specific behavior of the evaluated operator and input shape.

\subsection{Data Consolidation}
\label{subsec:data-consolidation}

After execution, a final stage of the script merged experimental data into a unified dataset.
The consolidation included:

\begin{itemize}
    \item \textbf{Peak Memory Values:} Extracted from each \texttt{.prof} file as the highest recorded \ac{RSS}.
    \item \textbf{Shape and Operator Metadata:} Inline, xline, and sample dimensions, as well as the applied seismic operator and associated attributes.
    \item \textbf{Derived Features:} Calculated values such as total volume (\text{inlines} $\times$ \text{xlines} $\times$ \text{samples}), logarithmic transformations, and geometric descriptors including surface area and diagonal length.
\end{itemize}

This merged dataset was stored in structured \texttt{.csv} files under \texttt{OUTPUT\_DIR}, forming the foundation for subsequent modeling and analysis.

\subsection{Repository Structure and Notebooks}
\label{subsec:repository-structure-and-notebooks}

The entire experiment, including code, automation scripts, and analytical notebooks, is publicly available:
\begin{itemize}
    \item \textbf{\texttt{Github repository}}\cite{delucca2025experiment2} containing all source code and setup instructions.
    \item \textbf{\texttt{scripts/experiment.sh}}\cite{delucca2025experiment2script} for end-to-end execution of data generation, processing, and profiling.
    \item \textbf{\texttt{notebooks/}}~\cite{delucca2025experiment2notebooks} with Jupyter notebooks covering data exploration, model training, hyperparameter optimization, and error diagnostics.
\end{itemize}

This structure ensures full reproducibility.
Users can extend the experiment by modifying input shapes, seismic operators, or container settings without altering the orchestration logic.
All components—containers, execution scripts, and analytical workflows—remain tightly integrated to support extensibility and transparent evaluation.